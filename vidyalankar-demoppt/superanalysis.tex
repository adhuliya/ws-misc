\begin{frame}
  \frametitle{Super-Analyses}

  \begin{block}{Principle}
      Analyses can be combined together by designing special flow functions, over their product domains [\prettyref{bib:cousot1979systematic}].
  \end{block}

  Some super-analyses are,
    \begin{itemize}
        \item Conditional Constant Propagation [\prettyref{bib:wegman1991constant}]
        \item Constant Propagation and Pointer Analysis [\prettyref{bib:pioli1999combining}]
        \item LFCPA [\prettyref{bib:khedker2012liveness}]
    \end{itemize}

\end{frame}


\begin{frame}
  \frametitle{Super-Analyses : Advantages}

    \begin{itemize}
        \item They can exploit the best possible interactions between analyses.
        \item They provide the most precise results.
        \item They solve the ubitquitous phase ordering problem (discussed separately).
    \end{itemize}
\end{frame}


\begin{frame}
  \frametitle{Super-Analyses : Disadvantages}

    \begin{itemize}
        \item They have to be manually designed.
        \item They combine few analyses (generally two or three).
        \item It is very hard, if not impossible, to manually design a super analysis that comprises of all known analyses.
    \end{itemize}

\end{frame}


%\begin{frame}
%  \frametitle{title}
%\end{frame}


